\documentclass[14pt]{article}

\usepackage{graphicx}

\begin{document}
\title{A REPORT ABOUT THE PRICES OF DIFFERENT COMMODITIES FOR THE LAST TWO YEARS.}
       \author{ ARYATWIJUKA JUDITH.   REG NO.15/U/3718/PS. ST NO.215004216}

\date{\today}
\maketitle
\tableofcontents

\section{ACKNOLEDGEMENT:}
           I would like to sincerely thank all the venders who accepted to share information with me about the prices of different commodities, challenges, even about personal issues at times.
           I also take the honor to thank my lecturer Mr. Ernest Mwebaze for the technology he has introduced me to and for this project which has enriched me with new knowledge about the current technologies and better methods of research.
\section{PROCEDURE:}
I followed a series of procedures to come up with this project:\par
1)	Downloading ODK collect on my android phone.\par
2)	Logged in into my Google account and searched for “Google cloud platform” using the Google search followed some steps and created a one ODK project.\par
3)	Downloaded the ODK aggregate on my PC.\par
4)	I used it to configure an aggregate server which was successfully configured.\par
5)	Then I created an xml form using the ODK build which I I transferred to my app on the phone.\par
6)	Interviewed venders.\par
7)	Then uploaded my results to the server.\par
\section{ABSTRACT:}
The aim of this report is to come up with the trends of prices of different commodities especially those that are mostly used and are necessities to people for the last two years which will help the people and the nation at large to plan well for the coming years(future).
\section{INTRODUCTION:}
                       According to my research, the prices of most commodities have generally been increasing with a percentage increment of averagely 40%.eg
\begin{table}[h]
\centering
\begin{tabular}{c c c}
\hline
COMMODITY &	PRICE(2016) &	CURRENT PRICE(2017)\\[0.5ex]
\hline
SUGAR &	4000	&7000\\
MAIZE FLOUR &	1200 &	2200\\
RICE &	3000 &	4500\\
\hline
\end{tabular}
\end{table}

Even when the prices are increasing, there are some commodities that have maintained their prices e.g. soap, water, salt, toothpaste etc.
\section{OBJECTIVES:}
\subsection{Main objective:}
    The main objective of this research is to make people be aware of the past financial status of the country, the present financial status so that they can plan well for the future.
\subsection{Specific objectives:}
a.	The government to devise other ways of caring for people especially those who get low incomes and cannot keep up their families.\par
b.	For people to study and understand the situation we are in so that they can work had for their families and the country at large.\par

\section{FUNCTIONALITY AND SCREEN SHOTS:}
This electronic data collection system helps in collecting data electronically and these are some of the screenshots of its output and the form used.

\begin{figure}[h!]
\includegraphics[width=100mm,scale=0.5]{1.jpg}
\caption{ODK aggregate server.}
\label{figure1}
\end{figure}

\begin{figure}[h!]
\includegraphics[width=100mm,scale=0.5]{3.jpg}
\caption{odk form.}
\label{figure2}
\end{figure}
\section{CHALLENGES:}
         While collecting this data and working on this project, I have met so many challenges and the main ones are;\par
i.	Limited access of internet: The internet in Makerere University is poor and this made it hard for me to do my project effectively since buying of internet is also expensive. This led to the delay of completion of my project.\par
ii.	Fixed schedule: I have had other course units to work on which have also consumed a lot of my time hence affecting the time I work on my project especially when I have tests which I must give maximum concentration.\par
iii.	Shortage of equipment to use: This project required me to use both a smartphone and a laptop which was somehow hard for me since I didn’t not own some of this which were costly.\par


\section{CONCLUSION:}
      Generally the trend of prices of commodities is always increasing and this has affected the economy of Uganda and the money has generally lost value. 
     This has affected the both the venders and the buyers equally. The buyers have foregone some of the commodities whose prices have been hiked and this this low demand has led the venders into losses since most of the prices end up expiring.
In conclusion I would request the government to do something about this because the Uganda shilling is seriously loosing value.

\begin{thebibliography}{10}

\bibitem{lateGuide}The venders

Available at \texttt{The community around kikoni}

\bibitem{latexGuide}The shopkeepers

Available at\texttt{Areas around kikumi mikumi and wadegeya}

\end{thebibliography}

\end{document}
